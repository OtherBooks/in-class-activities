\documentclass[answers]{exam}

%% Language and font encodings
\usepackage[english]{babel}

\usepackage[T1]{fontenc}

%% Sets page size and margins
\usepackage[a4paper,margin=2cm]{geometry}

%% Useful packages
\usepackage{amsmath, hyperref}
\usepackage{graphicx}
\usepackage{paralist, pgfplots}
%\setlength\FrameSep{4pt}






\pdfpagewidth 8.5in
 \pdfpageheight 11in


%General













%Theorem Environments

\begin{document}


\begin{center}
	\hfill Sandi Xhumari \\ \textbf{$\triangleright$ In-class Activity More Applications 5.4-5.5 $\triangleleft$}\\
\end{center}

\textbf{Purpose:} To use integrals in determining center of mass (weighted average), certain special volumes, and election winners.     \\

\textbf{Knowledge/Skills and Criteria for Success:} After this activity you should be able to

\begin{enumerate}[A.]
	\item find the center of mass.
	\item find volumes using tubes(shells).
	\item find election winner.
	\item 
	\item
	
	
\end{enumerate}

\textbf{Task:} 


\begin{questions}
	
\question Imagine you got 80\%, 90\% and 100\% respectively on the first 3 quizzes in a class. Each quiz is worth 20, 30, and 25 points respectively. What's your quiz grade so far?

\hfill \break
\hfill \break
\hfill \break
\hfill \break
\hfill \break

\question The center of mass of an object is intuitively the location where the object will balance without tilting. Imagine there's a thin triangle board of uniform weight distribution with vertices at locations $P_1 = (1, 7)$, $P_2 = (0, -5)$, and $P_3 = (-4, -7)$. You place objects of weights $m_1 = 1$, $m_2 = 3$, $m_3 = 2$ respectively on those vertices. Where is the center of mass of the triangle now?

\hfill \break
\hfill \break
\hfill \break
\hfill \break
\hfill \break
\hfill \break
\hfill \break
\hfill \break
\hfill \break
\hfill \break
\hfill \break

\question The centroid of a region is the center of mass of that region assuming uniform weight distribution. Given $f(x) \geq g(x)$ on an interval $[a, b]$, the centroid $(\bar{x}, \bar{y})$ of the region between them is given by the formulas:

\begin{enumerate}[(a)]
	\item $\displaystyle \bar{x} = \frac{\int_{a}^{b}x(f(x)-g(x))dx}{\int_{a}^{b}f(x)-g(x) dx}$, $\displaystyle \bar{y} = \frac{\int_{a}^{b}f^2(x)-g^2(x)dx}{2\int_{a}^{b}f(x)-g(x) dx}$.
	\item $\displaystyle \bar{x} = \frac{\int_{a}^{b}x(f^2(x)-g^2(x))dx}{\int_{a}^{b}f(x)-g(x) dx}$, $\displaystyle \bar{y} = \frac{\int_{a}^{b}f^2(x)-g^2(x)dx}{2\int_{a}^{b}f(x)-g(x) dx}$.
	\item $\displaystyle \bar{x} = \frac{\int_{a}^{b}x(f(x)-g(x))dx}{\int_{a}^{b}f(x)-g(x) dx}$, $\displaystyle \bar{y} = \frac{\int_{a}^{b}f^2(x)+g^2(x)dx}{2\int_{a}^{b}f(x)-g(x) dx}$.
	\item none of the above
\end{enumerate}

Moment of slice for $\bar{x}$: \hspace{2in} Moment of slice for $\bar{y}$: \\

Riemann Sum:  \hspace{2.44in} Riemann Sum:\\

Definite Integral: \hspace{2.3in} Definite Integral:\\

\question What is the centroid of the region bounded by the graphs $y = 4\sin(x)$, $y = x/6$, and $x = \pi/2$ touching the origin.



\hfill \break
\hfill \break
\hfill \break
\hfill \break
\hfill \break
\hfill \break
\hfill \break
\hfill \break
\hfill \break
\hfill \break


\question Suppose we take the area under the graph of $y = f(x)$ on $[a ,b]$ and we rotate it around the $y$-axis. Its volume is

\begin{enumerate}[(a)]
	\item $\displaystyle \int_a^b \pi f(x)^2dx$
	\item $\displaystyle \int_a^b 2\pi x f(x)dx$
	\item $\displaystyle \int_a^b 2\pi f(x)dx$
	\item None of the above.
\end{enumerate}

\question What is the volume of the area between $y = \sin(x)+2$ on $[0, 2\pi]$ rotated around the line $x = -1$?

\hfill \break
\hfill \break
\hfill \break
\hfill \break
\hfill \break
\hfill \break
\hfill \break
\hfill \break
\hfill \break
\hfill \break

\question Create a graph of the number of supporters of a certain political issue (education funding or taxes for instance). If there's two candidates running for office and they want to get the most votes, where should they position themselves to get the best chances to win? What if there's 3 candidates?

\end{questions}



\end{document}