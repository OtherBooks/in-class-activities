\documentclass[answers]{exam}

%% Language and font encodings
\usepackage[english]{babel}

\usepackage[T1]{fontenc}

%% Sets page size and margins
\usepackage[a4paper,margin=2cm]{geometry}

%% Useful packages
\usepackage{amsmath, hyperref}
\usepackage{graphicx}
\usepackage{paralist, pgfplots}
%\setlength\FrameSep{4pt}






\pdfpagewidth 8.5in
 \pdfpageheight 11in


%General













%Theorem Environments

\begin{document}


	\begin{center}
	\hfill Sandi Xhumari \\ \textbf{$\triangleright$ In-class Activity 8.5-8.6 $\triangleleft$}\\
\end{center}

\textbf{Purpose:} To learn how to carry out trigonometric substitutions in order to solve harder integrals. \\

\textbf{Knowledge/Skills and Criteria for Success:} After this activity you should be able to

\begin{enumerate}[A.]
	\item solve integrals involving a square root of a sum or difference of squares using trigonometric substitution.
	\item solve integrals involving trigonometric functions.
	\item 
	\item
	
\end{enumerate}

\textbf{Task:}


\begin{questions}
	
\question Recall from precalculus the Pythagorean identities and the double angle formulas involving trigonometric functions below.

\vspace{2in}


\question Consider using trigonometric substitution whenever you see a sum or a difference of squares since you could create a right triangle using the given sides and relate them to one of the angles using a trigonometric function. Practice by solving the following integrals.

\begin{enumerate}[(a)]
	
	\item $\displaystyle \int \frac{\sqrt{16x^2-49}}{x^4} dx =$
	
	\hfill \break
	\hfill \break
	\hfill \break
	\hfill \break
	\hfill \break
	\hfill \break
	\hfill \break
	\hfill \break
	\hfill \break
	\hfill \break
	\hfill \break
	\hfill \break
	
	\item $\displaystyle \int \frac{x}{\sqrt{100+4x^2}}dx = $
	
	\hfill \break
	\hfill \break
	\hfill \break
	\hfill \break
	\hfill \break
	\hfill \break
	\hfill \break
	\hfill \break
	\hfill \break
	\hfill \break
	\hfill \break
	\hfill \break
	
	
	\item $\displaystyle \int \frac{\sqrt{25-z^8}}{z^5} dx =$ 
	
	\hfill \break
	\hfill \break
	\hfill \break
	\hfill \break
	\hfill \break
	\hfill \break
	\hfill \break
	\hfill \break
	\hfill \break
	\hfill \break
	
	
\end{enumerate}

\question Determine the trignonometric substitution that can help solve integrals involving $X^2+A^2$, $X^2-A^2$, $A^2-X^2$, respectively. Draw a triangle for each and find $dx$.

\vspace{3in}

\question Why is the area of a circle of radius $R$ equal to $\pi R^2$? We have seen several proofs of this fact before in this course. Here's another possible argument: The equation of a circle of radius $R$ around the origin is $x^2+y^2=R^2$, so the upper half of it has equation $y = \sqrt{R^2-x^2}$. Find the following integral using a trigonometric substitution, and then use the result to deduce the area of a circle of radius $R$.\\

$\displaystyle  \int \sqrt{R^2-x^2} dx = $

\vspace{2in}

\newpage


\question Derive the following formulas we have used before using trigonometric substitution.

\includegraphics[scale=0.3]{AntiDerivativeInverseTrig}


\vspace{4in}


\question Solve the following integrals involving powers of $\sin(x)$ or $\cos(x)$ and note the general patterns. Feel free to put into your notecard these patterns for the challenge.
\begin{enumerate}[(a)]
	
	\item $\displaystyle \int \sin^4(x) dx =$
	
	\vspace{2in}

	\newpage
	
	
	\item $\displaystyle \int \cos^3(x) dx =$
	
	\hfill \break
	\hfill \break
	\hfill \break
	\hfill \break
	\hfill \break
	\hfill \break
	\hfill \break
	\hfill \break
	\hfill \break
	\hfill \break
	\hfill \break
	\hfill \break
	
	
	\item $\displaystyle \int \sin^2(x)\cos^2(x) dx =$ 
	
	\hfill \break
	\hfill \break
	\hfill \break
	\hfill \break
	\hfill \break
	\hfill \break
	\hfill \break
	\hfill \break
	\hfill \break
	\hfill \break
	\hfill \break
	\hfill \break
	\hfill \break
	\hfill \break
	\hfill \break
	\hfill \break
	\hfill \break
	
	

	
\end{enumerate}

\question Sometimes we face integrals with multiple angles so the following trig formulas are helpful. Use them to solve the following indefinite integral.

 \[\sin(\alpha)\cos(\beta) = \frac{1}{2}(\sin(\alpha + \beta) + \sin(\alpha-\beta))\]  \[\sin(\alpha)\sin(\beta) = \frac{1}{2}(\cos(\alpha - \beta) - \cos(\alpha+\beta))\]
\[\cos(\alpha)\cos(\beta) = \frac{1}{2}(\sin(\alpha + \beta) + \cos(\alpha-\beta))\]


$\displaystyle \int \sin(x)\cos(3x) dx =$ 

\newpage

\question Develop the rules to solve integrals of the form $\displaystyle \int \tan^n(x)\sec^m(x) dx$ for various non-negative integers $n$ and $m$. Start with pure powers of $\tan(x)$ and $\sec(x)$ separately. What happens when you do a substitution with $u = \tan(x)$ or $u = \sec(x)$? Focus on the big patterns.

\newpage

\question Google: ``The world's sneakiest substitution is undoubtedly the tangent half-angle technique." This is what is used to find a the formula for antiderivatives of secant and cosicant in the previous formulas we had. This technique though works with trig integrals involving rational functions of sine and cosine.
\begin{enumerate}[(a)]
	
	\item $\int \sec(x) dx = $
	
	
	\hfill \break
	\hfill \break
	\hfill \break
	\hfill \break
	\hfill \break
	\hfill \break
	\hfill \break
	\hfill \break
	\hfill \break
	\hfill \break
	\hfill \break
	\hfill \break
	\hfill \break
	\hfill \break
	\hfill \break
	\hfill \break
	
	\vfill
	
	\item $\int \frac{1}{2+\cos(x)} dx =$
	
	\hfill \break
	\hfill \break
	\hfill \break
	\hfill \break
	\hfill \break
	\hfill \break
	\hfill \break
	\hfill \break
	\hfill \break
	\hfill \break
	\hfill \break
	\hfill \break
	\hfill \break
	
\end{enumerate}


\end{questions}


\end{document}