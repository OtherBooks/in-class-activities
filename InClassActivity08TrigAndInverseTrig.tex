\documentclass[answers]{exam}

%% Language and font encodings
\usepackage[english]{babel}

\usepackage[T1]{fontenc}

%% Sets page size and margins
\usepackage[a4paper,margin=2cm]{geometry}

%% Useful packages
\usepackage{amsmath, hyperref}
\usepackage{graphicx}
\usepackage{paralist, pgfplots}
%\setlength\FrameSep{4pt}






\pdfpagewidth 8.5in
 \pdfpageheight 11in


%General













%Theorem Environments

\begin{document}


	\begin{center}
	\hfill Sandi Xhumari \\ \textbf{$\triangleright$ In-class Activity 7.1-7.3 $\triangleleft$}\\
\end{center}

\textbf{Purpose:} To review inverse functions, and their derivatives. Then use this knowledge to determine new integral formulas and combine them with substitution.  \\

\textbf{Knowledge/Skills and Criteria for Success:} After this activity you should be able to

\begin{enumerate}[A.]
	\item find anti-derivatives of functions related to trigonometric functions. 
	\item use substitution related to trigonometric functions.
	\item 
	\item A
	
\end{enumerate}

\textbf{Task:}


\begin{questions}


\question \textbf{Definitions.} A function $f$ is \textbf{one-to-one} (injective) if for every output $b$ there's exactly one input $a$, i.e., the equation $f(x) = b$ has a unique solution $x = a$ for every $b$ in the range of $f$. Graphically, if every horizontal line intersects the graph $y = f(x)$ in at most one point, then $f$ is one-to-one.\\
 In this case, there exists an inverse function denoted by $f^{-1}$ that undoes $f$, i.e., the equation $f(x) = b$ has a unique solution $x = f^{-1}(b)$ for every $b$ in the range of $f$. This implies that $f\circ f^{-1}(x) = x$ and $f^{-1}\circ f(x) = x$ for any $x$ in the appropriate domain. Algebraically, to find the inverse you swap $x$ and $y$, and solve for $y$. Graphically, we can determine the inverse function by reflecting about $y = x$. Find the inverse of $f(x) = 3\arctan(5x-1)+2$ both algebraically and graphically. Then use it to solve the following equations.
 
\vspace{2in }

 \[ 3\arctan(5x-1)+2 = 5 \hspace{1in} 3\arctan(5x-1)= \pi \]

\vspace{1in}

\question Solve the following integrals. You may want use and insert into your notecards the following formulas.

\includegraphics[scale=0.4]{AntiDerivativeTrig}\\

\includegraphics[scale=0.4]{AntiDerivativeInverseTrig}

\begin{enumerate}[(a)]
\item $\displaystyle \int \frac{5x}{1+x^2} = $

\hfill \break
\hfill \break
\hfill \break
\hfill \break
\hfill \break
\hfill \break

\item $\displaystyle \int_3^5 \frac{1}{1+x^2} = $

\hfill \break
\hfill \break
\hfill \break
\hfill \break
\hfill \break
\hfill \break

\item $\displaystyle  \int \frac{5x+1}{1+x^2} = $

\hfill \break
\hfill \break
\hfill \break
\hfill \break
\hfill \break
\hfill \break
\hfill \break

\item $\displaystyle  \int_{-\pi}^{\pi} \frac{1}{4+9x^2} = $

\hfill \break
\hfill \break
\hfill \break
\hfill \break
\hfill \break
\hfill \break
\hfill \break

\item $\displaystyle  \int \frac{1}{\sqrt{4-9x^2}} = $

\hfill \break
\hfill \break
\hfill \break
\hfill \break

\end{enumerate}



\end{questions}



\end{document}