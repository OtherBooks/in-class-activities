\documentclass[answers]{exam}

%% Language and font encodings
\usepackage[english]{babel}

\usepackage[T1]{fontenc}

%% Sets page size and margins
\usepackage[a4paper,margin=2cm]{geometry}

%% Useful packages
\usepackage{amsmath, hyperref}
\usepackage{graphicx}
\usepackage{paralist, pgfplots}
%\setlength\FrameSep{4pt}






\pdfpagewidth 8.5in
 \pdfpageheight 11in


%General













%Theorem Environments

\begin{document}




\begin{center}
 \hfill Sandi Xhumari \\ \textbf{$\triangleright$ In-class Activity 6.3 Applications to Differential Equations $\triangleleft$}\\
\end{center}


\textbf{Purpose:} To model various problems using differential equations, and realize how mathematics brings separate looking questions together.  \\





\textbf{Knowledge/Skills and Criteria for Success:} After this activity you should be able to
\begin{enumerate}[A.]
\item Model various application questions using differential equations.
\item Solve and interpret the solutions to differential equations in the context of the given application.
\item 
\item 

\end{enumerate}

\textbf{Task:} 

\begin{questions}
\question What do population, carbon dating, compound interest, and temperature have in common with one another?! Write ONE differential equation modeling all of the quantities in each of the following scenarios, and solve it. Under each part state what $x, y$ and $\frac{dy}{dx}$ stand for in the differential equation.

\begin{enumerate}[(a)]
	\item \textbf{Exponential Growth:} The number of births per year is proportional to the number of people in the population. 
	\item \textbf{Radioactive Decay/Carbon Dating:} The number of atoms per hour that release a particle is proportional to the number of atoms present. 
	\item \textbf{Compound Interest:} The number of dollars of interest per year is proportional to the amount of money in the bank account.
	\item \textbf{Newton's Law of Cooling:} The number of degrees the soup cools per minute is proportional to the temperature difference between the soup and the air. 
\end{enumerate}

\hfill \break
\hfill \break
\hfill \break
\hfill \break
\hfill \break

\question The number of bacteria on a Pietri plate $t$ hours after the experiment starts is proportional to the number of bacteria, with proportionality constant $r$. Assuming the experiment starts with $2000$ bacteria, and after one hour there's $2100$ bacteria, find the function modeling the number of bacteria, and then answer the following questions.

\hfill \break
\hfill \break
\hfill \break
\hfill \break
\hfill \break
\hfill \break
\hfill \break
\hfill \break
\hfill \break
\hfill \break

\begin{enumerate}[(a)]
	\item How many bacteria are there after $2$ hours?
	
	\hfill \break
	\hfill \break
	
	\item How long does it take for the number of bacteria to double?
	
	\hfill \break
	\hfill \break
	\hfill \break
	\hfill \break
	
	\item By what percentage are the number of bacteria growing each hour?
\end{enumerate}



\hfill \break
\hfill \break

\question Carbon dating is a method introduced by William Libby in 1946 in order to determine the age of the Dead Sea Scrolls by accounting for the radioactive decay of Carbon-14. He was only able to implement it in 1950 when the needed technology became available. Libby determined that a piece of linen from Qumran Cave 1 was about 167 BCE - 233 CE (google it for reference). Radioactive carbon–14 has a half–life of about 5700 years. What percent of carbon-14 was present in the piece of linen that Libby tested?

\vfill

\question An investment of \$25,100.00 earns 3.5\% annual interest, compounded continuously. If no funds are added or removed from this account, what is the future value of the investment after 15 years?

\vfill

\question Create an application question using Newton's Law of Cooling to catch a murderer by measuring the temperature of the body and using it to determine the time of death.

\vfill

\end{questions}



\end{document}