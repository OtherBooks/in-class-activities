\documentclass[answers]{exam}

%% Language and font encodings
\usepackage[english]{babel}

\usepackage[T1]{fontenc}

%% Sets page size and margins
\usepackage[a4paper,margin=2cm]{geometry}

%% Useful packages
\usepackage{amsmath, hyperref}
\usepackage{graphicx}
\usepackage{paralist, pgfplots}
%\setlength\FrameSep{4pt}






\pdfpagewidth 8.5in
 \pdfpageheight 11in


%General













%Theorem Environments

\begin{document}


	\begin{center}
	\hfill Sandi Xhumari \\ \textbf{$\triangleright$ In-class Activity 4.6 $\triangleleft$}\\
\end{center}

\textbf{Purpose:} To undo differentiation, i.e., find anti-derivatives. To find the derivative of a function it's a matter of following the definition, even if it's hard to apply the rules, but how about the reverse process, anti-differentiation? Suppose we record the velocity of your car throughout a given path. Is there a way to use this information to find out the location of the car at each moment in time? In other words, given the graph or equation of the velocity of an object, is there a way to obtain its position function (i.e., one of its anti-derivatives)? More generally, provided the rate of change of a quantity, is there a way to determine the value of the quantity in question? \\

\textbf{Knowledge/Skills and Criteria for Success:} After this activity you should be able to

\begin{enumerate}[A.]
	\item find anti-derivatives of common functions. 
	\item use substitution (change of variables) to find anti-derivatives. 
	\item 
	\item 
	
\end{enumerate}

\textbf{Task:}


\begin{questions}


\question Guess at least 3 explicit functions $f(x)$ such that the following hold, and then conjecture what are all the possible functions $f(x)$.
\begin{enumerate}[(a)]
\item $f'(x) = 2x$ 

\hfill \break
\hfill \break
\hfill \break


\item $f'(x) = 2\sin(x)\cos(x)$

\hfill \break
\hfill \break
\hfill \break
\hfill \break
\hfill \break
\hfill \break
\hfill \break
\hfill \break
\hfill \break

\end{enumerate}

\question Definitions of antiderivative and indefinite integral: Let $f(x)$ be a given function. An \textbf{antiderivative} of $f(x)$ is a function $F(x)$ such that $F\rq{}(x) = f(x)$. In general, there\rq{}s infinitely many antiderivatives of $f(x)$, so the set of all antiderivatives of $f(x)$ is the \textbf{indefinite integral}, denoted by
\[\int f(x)dx.\] 
The FTC says that the signed area function $A(x) = \int_{a}^{x}f(t)dt$ is an anti-derivative of $f(x)$ since $A'(x) = f(x)$. Thus, for any constants $a$ and $C$, we have

\[\int f(x)dx = \int_a^x f(x)dx + C,\]
which explains the similar notation. Write your conjectures using the indefinite integral notation.
\begin{enumerate}[(a)]
	\item $\displaystyle \int 2x dx =$ 
	
	\hfill \break
	
	
	\item $\displaystyle \int 2\sin(x)\cos(x) dx = $
	
	
\end{enumerate}

\question If $f\rq{}(x) = 0$ for all $x$ in the interval $(-\infty, 0)\cup(0, \infty)$, then $f(x) = C$ for some constant $C$ for all $x$ in the interval $(-\infty, 0)\cup(0, \infty)$. 
\begin{enumerate}[(a)]
\item True, and very confident.
\item True, but not confident.
\item False, and very confident.
\item False, but not confident.
\end{enumerate}

\question \textbf{Theorem.} If $F(x)$ and $G(x)$ are anti-derivatives of $f(x)$ defined on an \textbf{interval}, then there exists a constant $C$ such that $G(x) = F(x) + C$ on that \textbf{interval}. Thus, for any constant $C$ we have
\[ \int f(x) dx = F(x) + C. \]
\emph{Proof.} Let $F(x)$ and $G(x)$ be anti-derivatives of $f(x)$ defined on an interval I, and set $H(x) = G(x)-F(x)$. It is enough to show that $H(a) = H(b)$ for any numbers $a$ and $b$ in I, because

\hfill \break
\hfill \break
\hfill \break
\hfill \break
\hfill \break
\hfill \break

\question Write down in your notesheets for the next challenge all the derivatives and antiderivative rules below:


\includegraphics[scale=0.6]{AntiDerivativeRules}


 Use the above rules to find the following indefinite integrals.
\begin{enumerate}[(a)]
\item $\displaystyle \int 3x^2 - 4x + 5 + e^{2}dx  =$



\item $\displaystyle \int 7\sin(x) -3e^x + \frac{1}{\sqrt{x}}dx = $


\end{enumerate}

\question If $f$ is continuous on the interval $[a, b]$, then $\frac{d}{dx}\left(\int_a^b f(x) dx\right) = f(x)$.
\begin{enumerate}[(a)]
	\item True, and very confident.
	\item True, but not confident.
	\item False, and very confident.
	\item False, but not confident.
\end{enumerate}

\question Guess the following anti-derivatives if possible.
\begin{enumerate}[(a)]

\item $\displaystyle \int -2xe^{-x^2} dx = $


\item $\displaystyle  \int \frac{\ln(x)}{x} dx =$


\item $\displaystyle  \int \frac{7x}{-3x+1} dx =$



\end{enumerate}

\question Substitution (change of variables) allows us to figure out the above questions in a systematic (brainless?!) way. Essentially SUBSTITUTION is REVERSE CHAIN RULE! Substitution takes a hard looking integral in the $x$-world and changes it to an easier problem in the $u$-world. Thus, you are not done until all $x$ disappear! \textbf{WARNING:} You may really perform any substitution you want, but only a few may give you an easier integral. If your integral looks harder than what you started with, then probably it wasn\rq{}t the right substitution to make and you should try something else. Solve the following problems using substitution.

\begin{enumerate}[(a)]
\item $\displaystyle \int -2xe^{-x^2} dx = $

\hfill \break
\hfill \break
\hfill \break


\item $\displaystyle  \int \frac{\ln(x)}{x} dx =$

\hfill \break
\hfill \break
\hfill \break

\item $\displaystyle  \int \frac{7}{-3x+1} dx =$

\hfill \break
\hfill \break
\hfill \break
\hfill \break


\item $\displaystyle  \int \frac{7x}{-3x+1} dx =$

\hfill \break
\hfill \break
\hfill \break
\hfill \break

\end{enumerate}

\question Use substitution to find the following  integrals
\begin{enumerate}[(a)]
	
	\item $\displaystyle \int_0^{\pi/3} 2\sin(x)\cos(x)dx = $ 
	
	\hfill \break
	\hfill \break
	\hfill \break
	\hfill \break
	\hfill \break
	\hfill \break
	
	\item $\displaystyle \int_{-3}^0 x\sqrt{x+3} dx =$ 
	
	\hfill \break
	\hfill \break
	\hfill \break
	\hfill \break
	\hfill \break
	\hfill \break
	
	\item $\displaystyle \int_{-\pi/4}^{\pi/4} \tan(x) dx =$ 
	
	\hfill \break
	\hfill \break
	\hfill \break
	\hfill \break
	\hfill \break
	\hfill \break
	\hfill \break
	\hfill \break
		
	\item $\displaystyle \int_{-1}^3 -2x(-x^2)^3 dx$ = 
	
	\hfill \break
	\hfill \break
	\hfill \break
	\hfill \break
	
\end{enumerate}

\question Come up with any integral you wish that can be solved using the substitution $u = \tan(x)$, and then solve it using it.

\hfill \break
\hfill \break
\hfill \break
\hfill \break
\hfill \break
\hfill \break

\question Can you guess or use substitution to solve the following integral?

$\displaystyle \int e^{-x^2} dx =  $

\hfill \break
\hfill \break
\hfill \break

\question The function $y = e^{-x^2}$ is in fact the famous ``bell curve" function in statistics up to a scalar multiple. The area under the ``bell curve" from $x = a$ to $x = b$ is the percentage of total population that falls within $a$ and $b$ standard deviations from the mean. The famous $p$-value used for hypothesis tests, is the area from $-\infty$ to some standard deviation $x$. The total area under the ``bell curve" function is 1, or $100\%$. It is thus imperative to know the exact area under this function for any interval $a$ to $b$. This is usually provided as a huge table of values at the back of your stats textbook. To make estimates easier without using the table, it's helpful to employ the ``Empirical Rule\footnote{Read this https://www.statisticshowto.datasciencecentral.com/empirical-rule-2/}):" 68-95-99.7, as well as the ``Range Rule," which says that one standard deviation is approximately $\frac{\text{Max}-\text{Min}}{4}$. Suppose we want to understand the GPA-s of BC students, and we poll a random sample. Draw the ``bell curve" for it, and determine how many students are within one standard deviation below the mean and 2 standard deviations above the mean. What does that mean?


\end{questions}



\end{document}